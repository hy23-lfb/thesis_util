\chapter{Formulas}
\label{formeln}
\thispagestyle{empty}

Viele, viele bunte Formeln....


\section{FFT-Algorithmus}

Am  Beispiel der 1D-DFT soll eine Variante des FFT-Algorithmus
an  dieser  Stelle  beschrieben werden: Basis-2-Algorithmus
mit Zerlegung im Zeitraum (engl. {\em decimation in time 
radix-2  algorithm}) \cite{Aach2005}.

 Zun"achst wird die Fouriertransformierte wie folgt umgeschrieben:
\begin{equation} 
 \label{eq21}
F(u) = \frac{1}{N} \sum \limits_{x=0}^{N-1}f(x) e^{ -j 2 \pi \frac{u x}{N} }
  = \frac{1}{N} \sum \limits_{x=0}^{N-1}f(x) W_{N}^{ux}
\textrm{,}
\end{equation}
\begin{equation}
\label{eq22}
\textrm{mit  }
  W_{N} = e^{ -j 2 \pi \frac{1}{N} }
\textrm{  und  $u=0,1,\dots,N-1$.}
\end{equation}

Unter  der Annahme, da"s $N$ eine Potenz zur Basis $2$ ist  und
$n$ und $L$ zur Menge der nat"urlichen Zahlen geh"oren, l"a"st sich
$N$ ausdr"ucken durch
\begin{equation}
\label{eq23}
N=2^{n}=2L
\textrm{.}
\end{equation}

Durch Substitution von Gl. \ref{eq23} in Gl. \ref{eq21} ergibt sich
\begin{align} 
\label{eq24}
F(u) &= \frac{1}{2L} \sum \limits_{x=0}^{2L-1}f(x) W_{2L}^{ux} \nonumber \\
  &= \frac{1}{2} \left [ \frac{1}{L} \sum \limits_{x=0}^{L-1}f(2x) W_{2L}^{u(2x)} +
     \frac{1}{L} \sum \limits_{x=0}^{L-1}f(2x+1) W_{2L}^{u(2x+1)} \right ]
\textrm{,}
\end{align}
und mit $W_{2L}^{2ux} = W_{L}^{ux}$ (aus Gl. \ref{eq22}) folgt schlie"slich
\begin{equation}
\label{eq25}
F(u) = \frac{1}{2} \left [ \frac{1}{L} \sum \limits_{x=0}^{L-1}f(2x) W_{L}^{ux}+
     \frac{1}{L} \sum \limits_{x=0}^{L-1}f(2x+1) W_{L}^{ux} W_{2L}^{u} \right ]
\textrm{.}
\end{equation}

In einem  n"achsten Schritt definiert man f"ur gerade und ungerade
Funk\-tions\-werte je eine Funktion $F(u)$
\begin{equation} 
\label{eq26}
F_{gerade}(u) = \frac{1}{L} \sum \limits_{x=0}^{L-1}f(2x) W_{L}^{ux} 
\textrm{ und}
\end{equation}
\begin{equation} 
\label{eq27}
F_{ungerade}(u) = \frac{1}{L} \sum \limits_{x=0}^{L-1}f(2x+1) W_{L}^{ux}
\textrm{ f"ur $u=0,1,\dots,L-1$.}
\end{equation}
Durch  Einsetzen  von Gl. \ref{eq26} und Gl. \ref{eq27}  reduziert  sich
Gl. \ref{eq25} zu
\begin{equation}
\label{eq28}
F(u) = \frac{1}{2} \left [ F_{gerade}(u) + F_{ungerade}(u) W_{2L}^{u} \right ]
\textrm{.}
\end{equation}
Mit $W_{L}^{u+L} = W_{L}^{u}$, $W_{2L}^{u+L} = -W_{2L}^{u}$ und den Gl.~\ref{eq25}
bis Gl.~\ref{eq28} ergibt sich
\begin{equation}
\label{eq29}
F(u+L) = \frac{1}{2} \left [ F_{gerade}(u) - F_{ungerade}(u) W_{2L}^{u} \right ]
\textrm{.}
\end{equation}

Somit  besteht  der  FFT-Algorithmus  in  der  sukzessiven
Anwendung  der Halbierung der diskreten Funktion $F(u)$.  Mit
Gl.  \ref{eq28} und Gl. \ref{eq29} liegen $N/2$-Punkt-Transformationen vor.
Diese  lassen  sich  wiederum in zwei Transformationen  der
halben  L"ange  aufteilen. Man erh"alt "ahnliche  Gleichungen,
nur  mit  dem Unterschied, da"s die Phasenverschiebung  sich
verdoppelt  hat.  Die  Halbierung mu"s konsequenterweise  f"ur  beide
Funktionen, gerade und ungerade, durchgef"uhrt werden.

Im letzten Schritt erh"alt man f"ur $\tilde{u}=0$ :
\begin{align}
F(0) &= \frac{1}{2} \left [ \tilde{F}_{gerade}(0) + \tilde{F}_{ungerade}(0)
         W_{2\tilde{L}}^{0} \right ] \nonumber \\
     &= f(0)+f(1)
\textrm{ und }\\
F(0+\tilde{L}) &= F(1) = \frac{1}{2} \left [ \tilde{F}_{gerade}(0) - 
         \tilde{F}_{ungerade}(0) W_{2\tilde{L}}^{0} \right ] \nonumber \\
     &=& f(0)-f(1),
\end{align}
da $W=1$ und $\tilde{L}=1$ sind.

Am  Beispiel f"ur  $N=4$ soll dies demonstriert werden. Nach  Gl.~\ref{eq28}
und Gl.~\ref{eq29} erh"alt man ($L=2$)
\begin{align}
\label{eq210}
F(0) &= \frac{1}{2} \left [ F_{gerade}(0) + F_{ungerade}(0) W_{4}^{0} \right ]
\textrm{,} \\
F(2) &= \frac{1}{2} \left [ F_{gerade}(0) - F_{ungerade}(0) W_{4}^{0} \right ]
\textrm{,} \\
F(1) &= \frac{1}{2} \left [ F_{gerade}(0) + F_{ungerade}(0) W_{4}^{1} \right ]
\textrm{,} \\
F(3) &= \frac{1}{2} \left [ F_{gerade}(0) - F_{ungerade}(0) W_{4}^{1} \right ]
\label{eq211}
\textrm{.}
\end{align}
Mit Gl. \ref{eq22} gilt
\begin{equation}
  W_{4}^{0} = e^{ -j 2 \pi \frac{0}{4} } = 1
\textrm{ und }
  W_{4}^{1} = e^{ -j 2 \pi \frac{1}{4} } = -j
\textrm{.}
\end{equation}
Weiterhin nach \ref{eq26} und Gl. \ref{eq27}f"ur $u=0,1$
\begin{align}
F_{gerade}(0) &= \frac{1}{2} \sum \limits_{x=0}^{1}f(2x) W_{2}^{0}  \nonumber \\
  &= \frac{1}{2} \left [ f(0)W_{2}^{0} + f(2)W_{2}^{0} \right ]
  = \frac{1}{2} \left [ f(0) + f(2) \right ]
\textrm{,}\\
F_{ungerade}(0) &= \frac{1}{2} \sum \limits_{x=0}^{1}f(2x+1) W_{2}^{0} \nonumber \\
  &= \frac{1}{2} \left [ f(1)W_{2}^{0} + f(3)W_{2}^{0} \right ]
  = \frac{1}{2} \left [ f(1) + f(3) \right ]
\textrm{,}\\
F_{gerade}(1) &= \frac{1}{2} \sum \limits_{x=0}^{1}f(2x) W_{2}^{x} \nonumber \\
  &= \frac{1}{2} \left [ f(0)W_{2}^{0} + f(2)W_{2}^{1} \right ]
  = \frac{1}{2} \left [ f(0) - f(2) \right ]
\textrm{,}\\
F_{ungerade}(1) &= \frac{1}{2} \sum \limits_{x=0}^{1}f(2x+1) W_{2}^{x} \nonumber \\
  &= \frac{1}{2} \left [ f(1)W_{2}^{0} + f(3)W_{2}^{1} \right ]
  = \frac{1}{2} \left [ f(1) - f(3) \right ]
\textrm{,}
\end{align}
mit $W_{2}^{0} = 1$  und  $W_{2}^{1} = -1$.
Eingesetzt in Gl. \ref{eq210} bis Gl. \ref{eq211} folgt  das Ergebnis
\begin{align}
F(0) &= \frac{1}{2} 
  \left \{ 
    \frac{1}{2} \left [ f(0) + f(2) \right ] + \frac{1}{2} \left [ f(1) + f(3) \right ]
  \right \} \nonumber \\
     &= \frac{1}{4}
  \left \{ 
    \left [ f(0) + f(2) \right ] + \left [ f(1) + f(3) \right ]
  \right \}
\textrm{,}\\
F(2) &= \frac{1}{2} 
  \left \{ 
    \frac{1}{2} \left [ f(0) + f(2) \right ] - \frac{1}{2} \left [ f(1) + f(3) \right ]
  \right \} \nonumber \\
     &= \frac{1}{4}
  \left \{ 
    \left [ f(0) + f(2) \right ] - \left [ f(1) + f(3) \right ]
  \right \}
\textrm{,}\\
F(1) &= \frac{1}{2} 
  \left \{ 
    \frac{1}{2} \left [ f(0) - f(2) \right ] + \frac{1}{2} \left [ f(1) - f(3) \right ] (-j)
  \right \} \nonumber \\
     &= \frac{1}{4}
  \left \{ 
    \left [ f(0) - f(2) \right ] - j \left [ f(1) - f(3) \right ]
  \right \}
\textrm{,}\\
F(3) &= \frac{1}{2} 
  \left \{ 
    \frac{1}{2} \left [ f(0) - f(2) \right ] - \frac{1}{2} \left [ f(1) - f(3) \right ] (-j)
  \right \} \nonumber \\
     &= \frac{1}{4}
  \left \{ 
    \left [ f(0) - f(2) \right ] + j \left [ f(1) - f(3) \right ]
  \right \}
\textrm{.}
\end{align}
