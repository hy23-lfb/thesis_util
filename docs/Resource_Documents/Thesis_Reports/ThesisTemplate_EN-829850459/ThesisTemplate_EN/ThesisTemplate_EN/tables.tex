\chapter{Tables}
\label{Tabellen}
\thispagestyle{empty}

Wie erzeuge ich eine Tabelle?

Schöne Tabellen lassen sich mit dem Paket \textbf{booktabs} erzeugen, das auf die im Buchdruck unnötigen vertikalen Linien verzichtet.

% Results candidate segmentation comparison
\begin{table}[ht]\footnotesize%
\caption[]{Cell segmentation method comparison results. Performance and error measures for the segmentation of 40 test set images are provided, each method used optimized parameters.

\textit{ME}: misclassification error, \emph{MHD}: modified Hausdorff distance, \emph{OS}: over-segmentation, \emph{US}: under-segmentation, \emph{S}: sensitivity and \emph{PERF}: performance. The interquartile range (\emph{IQR}) is given for estimation of statistical dispersion of ME and MHD.}
\label{tab:candidate_segmentation_comparison}
\centering%
\begin{tabular}{p{3mm}p{4cm}*{8}{r}}
\toprule
% Header
& & \multicolumn{2}{c}{ME} & \multicolumn{2}{c}{MHD (\si{\micro\metre})} & \multicolumn{2}{c}{(\%)} & \multicolumn{2}{c}{} \\
%
\cmidrule(r){3-4} \cmidrule(r){5-6} \cmidrule(r){7-8}
%
& & \multicolumn{1}{c}{$\mu$} & \multicolumn{1}{c}{IQR} & \multicolumn{1}{c}{$\mu$} & \multicolumn{1}{c}{IQR} & \multicolumn{1}{c}{OS} & \multicolumn{1}{c}{US} & \multicolumn{1}{c}{S} & \multicolumn{1}{c}{PERF}\\
%
\midrule

                                  % method & misclass_err_mean & misclass_err_iqr & hausdorff_mean & hausdorff_iqr & oversegmentation & undersegmentation & sensitivity & performance \\
\multicolumn{10}{l}{Thresholding methods}\\
                 &  Otsu's selection method &            0.5009 &             0.68 &           10.3 &          17.6 &              0.0 &               9.4 &      0.9726 &       0.428 \\
      & Block-based selection, OD weighting &            0.3857 &             0.59 &            2.1 &           3.5 &              2.7 &               2.9 &      0.9315 &       0.503 \\
  & \textbf{Block-based, OD weighting, color distance} &            0.3929 &             0.53 &            2.8 &           4.4 &              0.0 &               4.5 &      0.9589 &       \textbf{0.533} \\
\multicolumn{10}{l}{Mean shift methods}\\
                         &  Marker stain OD &            0.4632 &             0.78 &            0.8 &           0.5 &              4.1 &               0.0 &      0.6027 &       0.187 \\
                          & Luv color space &            0.5278 &             0.79 &            2.2 &           2.7 &             12.3 &               0.0 &      0.7808 &       0.252 \\
                                   &  L-ODM &            0.4766 &             0.65 &            6.7 &           2.1 &             63.0 &               1.4 &      0.8767 &       0.147 \\
                                    &  L-OD &            0.4353 &             0.68 &            6.0 &           2.4 &             41.1 &               2.0 &      0.8356 &       0.228 \\
\bottomrule
\end{tabular}
\end{table}


\newpage

In   diesem   Abschnitt   sollen lediglich   die   grundlegenden
Eigenschaften  der  Fouriertransformation kurz tabellarisch dargestellt werden
(Tabelle  \ref{tableeigenft}). Die  Beweise zu den Regeln und  alle  Eigenschaften
sind in \cite{Aach2005} zu finden.

\begin{table}[!h]
 \begin{center}
  \begin{tabular}{|ccc|}
 \hline
 Ortsbereich &$\circ \hspace{-0.15cm} - \hspace{-0.15cm} \bullet$& Frequenzbereich\\\hline
 \hline
 Linearit"at &&  Linearit"at \\
 $k_{1}g(x)+k_{2}f(x)$ && $k_{1}G(u)+k_{2}F(u)$ \\\hline

 Symmetrie   &&  Symmetrie \\
 $F(x)$ && $f(-u)$ \\\hline

 Ortsskalierung && reziproke\\
 && Frequenzskalierung \\
 $f(kx)$ && $\frac{1}{|k|}F(\frac{u}{k})$\\\hline

 reziproke&&\\
 Ortsskalierung && Frequenzskalierung \\
 $\frac{1}{|k|}f(\frac{x}{k})$ && $F(ku)$\\\hline

 Ortsverschiebung && Phasenverschiebung \\
 $f(x-x_{0})$ && $F(u) e^{ -j 2 \pi u x_{0} }$\\\hline

 Modulation && Frequenzverschiebung \\
 $f(x) e^{ -j 2 \pi x u_{0} }$ && $F(u-u_{0})$\\\hline

 gerade Funktion && reelle Funktion \\
 $f_{g}(x)$ && $F_{g}(u)=R_{g}(u)$\\\hline

 ungerade Funktion && imagin"are Funktion \\
 $f_{u}(u)$ && $F_{u}(u)=jI_{u}(u)$\\\hline

 reelle Funktion && gerader Realteil,\\
 && ungerader Imagin"arteil \\
 $f(x)=f_{r}(u)$ && $F(u)=R_{g}(u)+j I_{u}(u)$\\\hline

 imagin"are Funktion && ungerader Realteil, \\
 && gerader Imagin"arteil \\
 $f(x)=j f_{i}(u)$ && $F(u)=R_{u}(u)+j I_{g}(u)$\\\hline

  \end{tabular}
 \end{center}
 \caption[Wichtige Eigenschaften der Fouriertransformation]{
Wichtige Eigenschaften der Fouriertransformation}
 \label{tableeigenft}
\end{table}
